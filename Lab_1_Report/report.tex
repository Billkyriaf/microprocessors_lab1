\documentclass[a4paper, 12pt]{report}

\usepackage[english]{babel}
\usepackage{hyphenat}
\usepackage{indentfirst}
\usepackage{geometry}
\usepackage{float}
\usepackage{listings, lstautogobble}
\usepackage{color}
\usepackage{graphicx}


% Hyphenation
\hyphenpenalty 10000
\exhyphenpenalty 10000

% Layout
\geometry{a4paper, left=25mm, right=25mm, top=30mm, bottom=20mm}

% Cover page
\title{\Large{\textbf{Microprocessors Lab1}}}
\author{Kyriafinis Vasilis 9797, Koro Erika 9707}
\date{April 9, 2022}


\begin{document}
    \maketitle
    
    \section*{Introduction}
    The objective of this assignment is to calculate the hash of a given string.
    The hash function is a given hash table, that matches capital letters to integers, implemented in ARM
    assembly and called by a main C routine. 

    \section*{Code description}
    In the main C routine a string is given in order to calculate its hash. 
    In the ARM assembly, hash function has three arguments: 

    \begin{enumerate}
        \item r0: Contains the hash table's memory address.
        \item r1: Contains the given string's memory address.
        \item r2: Contains result's memory address.
    \end{enumerate}

    In the beginning, the first string character is loaded from the memory and then is compared to the value
    90(The 'Z' in ASCII). If the character is bigger, it means that it is not a number nor a capital
    letter, so branch to not\_number tag, else compare it to the value 65(the 'A' in ASCII). If it is less than
    65, it means that it is not a capital letter and branches to not\_capital tag. If none of the comparisons is
    true, it means that the character is a capital letter.
    
    In order to know which number should be selected from 
    the hash table, the subtraction between the character and the 'A' character is used as an offset for the
    table. The hash table contains integers that are 32 bits, so the offset is multiplied with 4(shift left 2).
    Afterwards, the hash value is added to the register that holds temporarily the calculated hash value and finally
    the pointer, that holds the string, is increased to point to the next character of the string 
    to be examined as well. 

    When branching to the not\_capital tag the character is compared to 48(0 in ASCII) and then to 57(9 in ASCII).
    If the character is between those numbers it means that it is a number and its value is subtracted from the hashed
    value as needed. To get the value, from the character is subtracted the value 48 to convert it to an integer.
    Eventually, the string pointer increases to point to the next character and the program branches to loop tag.
    Otherwise, when the above if statement is false the program branches to the not\_number tag.

    When branching to the not\_number tag, it is checked whether the character is null, in order to exit and store 
    the calculated hash value else it continues to the next loop.
    
    \section*{Problems Encountered}

    In an effort to limit the usage of brach commands we tried to use conditional instructions. The program required
    to execute more than one conditional instructions in a row and an it block was needed. Due to the errors that 
    arose at run time (read permissions) the idea was abandoned.

    \section*{Testing and Debugging}

    For testing purposes a C function was created in order to confirm the hash results. The initial string was hardcoded
    to the code. Many different strings where tested. Also the printf function was called from the assembly function to 
    confirm the results along with the C function. 
    
    Regarding the debugging of the assembly code the Keil debugger was used. Especially useful was the memory and register viewer because it allowed 
    to check the validity if every instruction executed. 

    The memory view windows allows to check the content of the memory on a specific memory address in decimal, 
    hexadecimal and other formats. Finally the register panel displays the value of every register in hexadecimal
    format.

    \end{document}